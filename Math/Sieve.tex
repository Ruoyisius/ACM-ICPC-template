\subsection{各种筛}
\begin{lstlisting}[language=c++]
线性筛素数
保证每个数只会被它的最小质因子给筛掉(不同于埃氏筛中每个数会被它所有质因子筛一遍从而使复杂度过高)
int pri[N],tot,zhi[N];//zhi[i]为1的表示不是质数
void sieve()
{
    zhi[1]=1;
    for (int i=2;i<=n;i++)
    {
        if (!zhi[i]) pri[++tot]=i;
        for (int j=1;j<=tot&&i*pri[j]<=n;j++)
        {
            zhi[i*pri[j]]=1;
            if (i%pri[j]==0) break;
        }
    }
}
所有线性筛积性函数都必须基于线性筛素数。
线性筛莫比乌斯函数
int mu[N],pri[N],tot,zhi[N];
void sieve()
{
    zhi[1]=mu[1]=1;
    for (int i=2;i<=n;i++)
    {
        if (!zhi[i]) pri[++tot]=i,mu[i]=-1;
        for (int j=1;j<=tot&&i*pri[j]<=n;j++)
        {
            zhi[i*pri[j]]=1;
            if (i%pri[j]) mu[i*pri[j]]=-mu[i];
            else {mu[i*pri[j]]=0;break;}
        }
    }
}
线性筛欧拉函数
int phi[N],pri[N],tot,zhi[N];
void sieve()
{
    zhi[1]=phi[1]=1;
    for (int i=2;i<=n;i++)
    {
        if (!zhi[i]) pri[++tot]=i,phi[i]=i-1;
        for (int j=1;j<=tot&&i*pri[j]<=n;j++)
        {
            zhi[i*pri[j]]=1;
            if (i%pri[j]) phi[i*pri[j]]=phi[i]*phi[pri[j]];
            else {phi[i*pri[j]]=phi[i]*pri[j];break;}
        }
    }
}
线性筛约数个数
记d(i)
d(i)表示i的约数个数,d(i)=∏k (i=1)(ai+1)   d(i)=∏i=1k(ai+1)
维护每一个数的最小值因子出现的次数(即a1)即可
int d[N],a[N],pri[N],tot,zhi[N];
void sieve()
{
    zhi[1]=d[1]=1;
    for (int i=2;i<=n;i++)
    {
        if (!zhi[i]) pri[++tot]=i,d[i]=2,a[i]=1;
        for (int j=1;j<=tot&&i*pri[j]<=n;j++)
        {
            zhi[i*pri[j]]=1;
            if (i%pri[j]) d[i*pri[j]]=d[i]*d[pri[j]],a[i*pri[j]]=1;
            else {d[i*pri[j]]=d[i]/(a[i]+1)*(a[i]+2);a[i*pri[j]]=a[i]+1;break;}
        }
    }
}
\end{lstlisting}
