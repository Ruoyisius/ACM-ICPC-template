\subsection{线段树}
\begin{lstlisting}[language=c++]
#include<bits/stdc++.h>
using i64=long long;
template<class Info>
struct SegmentTree {
    int n;
    std::vector<Info> info;
    SegmentTree() : n(0) {}
    SegmentTree(int n_, Info v_ = Info()) {
        init(n_, v_);
    }
    template<class T>
    SegmentTree(std::vector<T> init_) {
        init(init_);
    }
    void init(int n_, Info v_ = Info()) {
        init(std::vector(n_, v_));
    }
    template<class T>
    void init(std::vector<T> init_) {
        n = init_.size();
        info.assign(4 << std::__lg(n), Info());
        std::function<void(int, int, int)> build = [&](int p, int l, int r) {
            if (r - l == 1) {
                info[p] = init_[l];
                return;
            }
            int m = (l + r) / 2;
            build(2 * p, l, m);
            build(2 * p + 1, m, r);
            pull(p);
        };
        build(1, 0, n);
    }
    void pull(int p) {
        info[p] = info[2 * p] + info[2 * p + 1];
    }
    void modify(int p, int l, int r, int x, const Info &v) {
        if (r - l == 1) {
            info[p] = v;
            return;
        }
        int m = (l + r) / 2;
        if (x < m) {
            modify(2 * p, l, m, x, v);
        } else {
            modify(2 * p + 1, m, r, x, v);
        }
        pull(p);
    }
    void modify(int p, const Info &v) {
        modify(1, 0, n, p, v);
    }
    Info rangeQuery(int p, int l, int r, int x, int y) {
        if (l >= y || r <= x) {
            return Info();
        }
        if (l >= x && r <= y) {
            return info[p];
        }
        int m = (l + r) / 2;
        return rangeQuery(2 * p, l, m, x, y) + rangeQuery(2 * p + 1, m, r, x, y);
    }
    Info rangeQuery(int l, int r) {
        return rangeQuery(1, 0, n, l, r);
    }
    template<class F>
    int findFirst(int p, int l, int r, int x, int y, F pred) {
        if (l >= y || r <= x || !pred(info[p])) {
            return -1;
        }
        if (r - l == 1) {
            return l;
        }
        int m = (l + r) / 2;
        int res = findFirst(2 * p, l, m, x, y, pred);
        if (res == -1) {
            res = findFirst(2 * p + 1, m, r, x, y, pred);
        }
        return res;
    }
    template<class F>
    int findFirst(int l, int r, F pred) {
        return findFirst(1, 0, n, l, r, pred);
    }
    template<class F>
    int findLast(int p, int l, int r, int x, int y, F pred) {
        if (l >= y || r <= x || !pred(info[p])) {
            return -1;
        }
        if (r - l == 1) {
            return l;
        }
        int m = (l + r) / 2;
        int res = findLast(2 * p + 1, m, r, x, y, pred);
        if (res == -1) {
            res = findLast(2 * p, l, m, x, y, pred);
        }
        return res;
    }
    template<class F>
    int findLast(int l, int r, F pred) {
        return findLast(1, 0, n, l, r, pred);
    }
};
 
constexpr i64 inf = 1E18;
 
struct Info {
    i64 cnt = 0;
    i64 sum = 0;
    i64 min = inf;
};
 
Info operator+(Info a, Info b) {
    Info c;
    c.cnt = a.cnt + b.cnt;
    c.sum = a.sum + b.sum;
    c.min = std::min(a.min, b.min);
    return c;
}
\end{lstlisting}

\subsection{线段树-数组实现}
\begin{lstlisting}[language=c++]
struct segmentTree{
    int mx[N<<2], num[N<<2], sum[N<<2];
    void update(int id) {
        int ls = id << 1;
        int rs = (id << 1) | 1;
        num[id] = num[ls] + num[rs];
        sum[id] = sum[ls] + sum[rs];
        mx[id] = max(mx[rs], mx[ls]);
        return;
    }
    void build(int l, int r, int id) {
        if (l == r) {
            mx[id] = a[l];
            num[id] = a[l] < 100;
            sum[id] = a[l];
            return;
        }
        int mid = (l + r) >> 1;
        build(l, mid, id << 1);
        build(mid + 1, r, (id << 1) | 1);
        update(id);
        return;
    }
    void change(int x, int y, int l, int r,int id) {
        if (mx[id] < 10) return;
        if (l == r) {
            a[l] = a[l] * 2 / 3;
            mx[id] =  a[l];
            sum[id] = a[l];
            num[id] = a[l] < 100;
            return;
        }
        int m = l + r >> 1;
        if (x <= m) 
            change(x, y, l, m, id << 1);
        if (y > m)
            change(x, y, m + 1, r, id << 1 | 1);
        update(id);
    }
    int querynum(int x, int y, int l, int r, int id) {
        if(x <= l && r <= y) 
            return num[id];
        int m = l + r >> 1;
        if (x <= m && y > m)
            return querynum(x, y, l, m, id << 1) + querynum(x, y, m +1, r, id << 1 | 1);
        else if (x <= m)
            return querynum(x, y, l, m, id << 1);
        else
            return querynum(x, y, m + 1, r, id << 1 | 1);
    }
    int querysum(int x, int y, int l, int r, int id) {
        if (x <= l && r <= y)
            return sum[id];
        int m = l + r >> 1;
        if (x <= m && y > m) 
            return querysum(x, y, l, m, id << 1) + querysum(x, y, m + 1, r, id << 1 | 1);
        else if (x <= m)
            return querysum(x, y, l, m, id << 1);
        else
            return querysum(x, y, m + 1, r, id << 1 | 1);
    }
}T;
signed main(void)
{
    std::ios::sync_with_stdio(false);
    std::cin.tie(nullptr);
    cin >> n >> m;
    for (int i = 1; i <= n; i++)
        cin >> a[i];

    T.build(1, n, 1);
    for (int i = 1; i <= m; i++) {
        int op, l, r; cin >> op >> l >> r;
        if (op == 1) {
            T.change(l, r, 1, n, 1);
        } else if (op == 2) {
            cout << T.querynum(l, r, 1, n, 1) << endl;
        } else {
            cout << T.querysum(l, r, 1, n, 1) << endl;
        }
    }
    return 0;
}
\end{lstlisting}