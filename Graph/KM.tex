\subsection{KM}
\begin{lstlisting}[language=c++]
const int N = 500;
ll w[N][N]; // 边权
ll la[N], lb[N], upd[N]; // 左、右部点的顶标
bool va[N], vb[N]; // 访问标记:是否在交错树中
ll match[N]; // 右部点匹配了哪一个左部点
ll last[N]; // 右部点在交错树中的上一个右部点,用于倒推得到交错路
int n;
const int INF = 1e15;
const int infll = INF;
bool dfs(ll x, ll fa, vector<vector<int>> &w) 
{
    va[x] = 1;
    for(int y = 1; y <= n; y++)
    {
        if(!vb[y])
        {
            if(la[x] + lb[y] == w[x][y]) 
            { 
                vb[y] = 1; last[y] = fa;
                if(!match[y] || dfs(match[y], y, w)) 
                {
                    match[y] = x;
                    return true;
                }
            }
            else if(upd[y] > la[x] + lb[y] - w[x][y]) 
            {
                upd[y] = la[x] + lb[y] - w[x][y];
                last[y] = fa;
            }
        }
    }
    return false;
}

ll KM(vector<vector<int>> & w) 
{
    for(int i = 1; i <= n; i++) 
    {
        la[i] = -infll;
        lb[i] = 0;
        for(int j = 1; j <= n; j++) la[i] = max(la[i], w[i][j]);
    }
    for(int i = 1; i <= n; i++) 
    {
        memset(va, 0, sizeof(va));
        memset(vb, 0, sizeof(vb));
        for(int j = 1; j <= n; j++) upd[j] = infll;
        // 从右部点st匹配的左部点match[st]开始dfs,一开始假设有一条0-i的匹配
        int st = 0; match[0] = i;
        while(match[st])    // 当到达一个非匹配点st时停止
        { 
            ll delta = infll;
            if(dfs(match[st], st, w)) break;
            for(int j = 1; j <= n; j++)
                if(!vb[j] && delta > upd[j]) 
                {
                    delta = upd[j];
                    st = j;         // 下一次直接从最小边开始DFS
                }
            for(int j = 1; j <= n; j++)  // 修改顶标
            {
                if(va[j]) la[j] -= delta;
                if(vb[j]) lb[j] += delta; 
                else upd[j] -= delta;
            }
            vb[st] = true;
        }
        while(st)   // 倒推更新增广路 
        { 
            match[st] = match[last[st]];
            st = last[st];
        }
    }
    ll ans = 0;
    for(int i = 1; i <= n; i++) 
        if(match[i]) {
            dbg(i, match[i], w[match[i]][i]);
            ans += w[match[i]][i];
        }
    return ans;
}

// struct
template <typename T>
struct hungarian {  // km
  int n;
  vector<int> matchx;  // 左集合对应的匹配点
  vector<int> matchy;  // 右集合对应的匹配点
  vector<int> pre;     // 连接右集合的左点
  vector<bool> visx;   // 拜访数组 左
  vector<bool> visy;   // 拜访数组 右
  vector<T> lx;
  vector<T> ly;
  vector<vector<T> > g;
  vector<T> slack;
  T inf;
  T res;
  queue<int> q;
  int org_n;
  int org_m;

  hungarian(int _n, int _m) {
    org_n = _n;
    org_m = _m;
    n = max(_n, _m);
    inf = numeric_limits<T>::max();
    res = 0;
    g = vector<vector<T> >(n, vector<T>(n));
    matchx = vector<int>(n, -1);
    matchy = vector<int>(n, -1);
    pre = vector<int>(n);
    visx = vector<bool>(n);
    visy = vector<bool>(n);
    lx = vector<T>(n, -inf);
    ly = vector<T>(n);
    slack = vector<T>(n);
  }

  void addEdge(int u, int v, int w) {
    g[u][v] = max(w, 0);  // 负值还不如不匹配 因此设为0不影响
  }

  bool check(int v) {
    visy[v] = true;
    if (matchy[v] != -1) {
      q.push(matchy[v]);
      visx[matchy[v]] = true;  // in S
      return false;
    }
    // 找到新的未匹配点 更新匹配点 pre 数组记录着"非匹配边"上与之相连的点
    while (v != -1) {
      matchy[v] = pre[v];
      swap(v, matchx[pre[v]]);
    }
    return true;
  }

  void bfs(int i) {
    while (!q.empty()) {
      q.pop();
    }
    q.push(i);
    visx[i] = true;
    while (true) {
      while (!q.empty()) {
        int u = q.front();
        q.pop();
        for (int v = 0; v < n; v++) {
          if (!visy[v]) {
            T delta = lx[u] + ly[v] - g[u][v];
            if (slack[v] >= delta) {
              pre[v] = u;
              if (delta) {
                slack[v] = delta;
              } else if (check(v)) {  // delta=0 代表有机会加入相等子图 找增广路
                                      // 找到就return 重建交错树
                return;
              }
            }
          }
        }
      }
      // 没有增广路 修改顶标
      T a = inf;
      for (int j = 0; j < n; j++) {
        if (!visy[j]) {
          a = min(a, slack[j]);
        }
      }
      for (int j = 0; j < n; j++) {
        if (visx[j]) {  // S
          lx[j] -= a;
        }
        if (visy[j]) {  // T
          ly[j] += a;
        } else {  // T'
          slack[j] -= a;
        }
      }
      for (int j = 0; j < n; j++) {
        if (!visy[j] && slack[j] == 0 && check(j)) {
          return;
        }
      }
    }
  }

  void solve() {
    // 初始顶标
    for (int i = 0; i < n; i++) {
      for (int j = 0; j < n; j++) {
        lx[i] = max(lx[i], g[i][j]);
      }
    }

    for (int i = 0; i < n; i++) {
      fill(slack.begin(), slack.end(), inf);
      fill(visx.begin(), visx.end(), false);
      fill(visy.begin(), visy.end(), false);
      bfs(i);
    }

    // custom
    for (int i = 0; i < n; i++) {
      if (g[i][matchx[i]] > 0) {
        res += g[i][matchx[i]];
      } else {
        matchx[i] = -1;
      }
    }
    cout << res << "\n";
    for (int i = 0; i < org_n; i++) {
      cout << matchx[i] + 1 << " ";
    }
    cout << "\n";
  }
};

\end{lstlisting}